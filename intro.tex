\section{What is diabetes mellitus?}
\paragraph{It is a condition}that prevents the affected person from controlling their blood glucose concentration by releasing insulin. This can occur for numerous reasons. As a result, diabetes can be categorised into two types: Type 1  and type 2 diabetes. These types of diabetes occur for different reasons, but have the same effect; both types increase the amount of blood glucose a person has.[1]
\subsection{Risks of diabetes}
\paragraph{If glucose}isn't removed from the bloodstream, cells cannot be fully functional as they don't respire as much as they need to. Next, if glucose builds up in the blood, low density lipoproteins are more likely to be deposited. This can lead to atheroma, which can clog up arteries causing strokes, angina and heart attacks. This a great risk, especially if you have type 2 diabetes.[1][7]
\paragraph{Another complication}is nerve damage-neuropathy-as capillaries that supply your nerves can get damaged by excess glucose. This is because the nerves can't get enough glucose to respire. This problem can lead to a loss of sensation in affected area and erectile dysfunction in men.[6]
\paragraph{Some other complications}include,firstly, damage to the feet caused by nerve damage and lack of blood flow can mean that even small cuts and blisters can become infected. As a result, they may need to be amputated. Also, diabetes can lead to an increased likelihood of the affected person having cataracts or glaucoma. Moreover, it can can cause kidney damage, meaning that the affected may require renal dialysis or a kidney transplant. This is because diabetics may not have glucose transporters that help with absorption of glucose,therefore the kidney is damaged by the excess glucose.[6]
\section{Blood glucose concentration}
\subsection{Euglycemia/Normoglycemia}
\paragraph{This is}when blood glucose concentration is at its normal level in the body. In most humans, this is about 90 $\textnormal{mg} cm^{-3}$ /4-6 $\textnormal{mmol} dm^{-3}$. Patients measure blood glucose concentration by pricking their finger with a lancing device after fasting, a large meal or large intake of carbohydrates.
\subsection{Hyperglycemia}
\paragraph{This is when}blood glucose concentration is very low.
